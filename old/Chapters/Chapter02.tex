\chapter{Introduzione ai sistemi operativi}
\begin{figure}[ht]
  \centering
  \incfig{struttura}
  \caption{Struttura base di un calcolatore}\label{fig:struttura}
\end{figure}

Lo scopo di questa sezione è di virtualizzare le componenti fisiche di un
calcolatore per renderne più facile l'utilizzo.

Sinonimi di \emph{sistema operativo} sono \textbf{macchina virtuale} (virtual
machine) e \textbf{gestore dlle risorse} (resource manager).

\section{Virtualizzazione della CPU}
L'unica nota in questo programma è l'utilizzo della funzione \emph{spin ()}, che
controlla l'orologio di sistema fino a quanto $1$ minuto non è passato, quindi
riprende l'esecuzione.

\lstinputlisting[caption=Interazione con la  CPU usando la funzione spin()]{listings/cpu.c}

Da notare che se il programma viene eseguito con un comando del tipo:
\begin{lstlisting}[language=bash,caption=Esecuzione in concorrenza]
prompt> ./cpu A & ./cpu B &./cpu C & ./cpu D
[7353]
[7354]
[7355]
[7356]
A
B
D
C
A
B
D
C
A
...
\end{lstlisting}

Possiamo osservare che sembra che vengano eseguiti simultaneamente quattro
programmi diversi, ciò è dovuto alla \emph{virtualizzazione della CPU}.

\section{Memoria virtualizzata}
\lstinputlisting[caption=Dimostrazione dell'utilizzo di indirizzi virtuali]{listings/mem.c}

La memoria \textbf{RAM} è generalmente virtualizzata, difatti se eseguiamo
il programma precedente con un comando tipo:
\begin{lstlisting}[language=bash,caption=Esecuzione in concorrenza]
prompt > ./mem &; ./mem &
[1] 24113
[2] 24114
(24113) address pointed by p: 0x200000
(24114) address pointed by p: 0x200000
(24113) p: 1
(24114) p: 1
(24113) p: 2
(24114) p: 2
(24113) p: 3
(24114) p: 3
(24113) p: 4
(24114) p: 4
...
\end{lstlisting}

Possiamo osservare il funzionamento della virtualizzazione della memoria,
infatti ogni programma è lanciato in un proprio \textbf{virtual address space};
semplifica la gestione della memoria per il programma ed è mappato ad una
sezione della memoria fisica.

\section{Periferiche}
Nell'interazione di un calcolatore con le periferiche, è il kernel che
interagisce direttamente con l' hardware, mentre le applicazioni possono
utilizzare le periferiche solo attraverso il kernel.


Vi sono $4$ diversi livelli di privilegi per interagire con i dispositivi:
\begin{itemize}
  \item $0$, il livello più privilegiato, utilizzato dal kernel;
  \item $1$ solitamente non utilizzato;
  \item $2$ solitamente non utilizzato;
  \item $3$ il livello meno provilegiato, utilizzato da tutti gli utenti
    compreso l'utente di \emph{root}.
\end{itemize}

Se l'utende richiede un interazione con una periferica, un comando di
\emph{TRAP} viene inviato al kernel, il quale restituise un comando di
\emph{return from TRAP}.

\section{Memoria}
\subsection{Spazi di indirizzamento}
Ad ogni processo in esecuzione corrisponde un \emph{address space}, una
virtualizzazione della memoria che comporta diversi vantaggi, tra cui non
permettere ai programmi di modificare memoria non nella propria \emph{address
space} ed in generale un efficienza maggiore.

Lo spazio di indirizzamento non può essere più grande della memoria fisica ed
occupa una sezione di memoria contigua.

Utilizzando questo metodo abbiamo due problemi di frammentazionem interna ed
esterna.


Un processo è composto da:
\begin{itemize}
  \item \textbf{codice:} statico, regione fissa;
  \item \textbf{dati:} regione fissa;
  \item \textbf{stack:} regione dinamica, può crescere o diminuire;
  \item \textbf{heap:} regione dinamica.
\end{itemize}


Quindi l'address space è composto, in ordine, da:
\begin{itemize}
  \item \textbf{codice del programma:} dove risiedono le istruzioni;
  \item \textbf{heap:} contiene i dati creati con i \emph{malloc} e le
    strutture dinamiche, cresce verso il basso;
  \item \textbf{memoria libera:} che può essere occupata dallo heap e dallo
    stack;
  \item \textbf{stack:} dove risiedono le variabili locali, i valori di ritorno
    delle funzioni, etc. e cresce verso l'alto.
\end{itemize}


All'interno del sistema operativo è contenuta una struttura dati che contiene
informazioni relative ai processi in memoria, \textbf{PCB} (Process Control
Bloc).

\subsection{Virtualizzazione}
Gli obiettivi della virtualizzazione sono: trasparenza, efficienza e protezione
ovvero un programma non deve accorgersi che sta utilizzando solo una porzione
della memoria disponibile.


Per implementare queste funzioni si utilizza una \textbf{MMU}; al suo interno
salviamo un registro chiamato base, il quale cambia a seconda del programma
in esecuzione.

Questa implementazione non è corretta in quanto non abbiamo indicato un limite
al valore che possiamo assegnare alla base, quindi i processi potrebbero
accedere a memoria successiva alla loro, per cui sono gestiti anche i limiti
su quali valori posso assegnare al registro base.

Nel caso di un \textit{segmentation fault} (errore relativo all'accesso di
memoria non propria) vengono sollevati gli \textbf{interrupt handlers}, e gli
\textbf{exception handlers}, i quali gestiscono le eccezioni.
Altri errori simili che sollevano un interrupt sono la divisione per $0$.

Gli interrupt possono anche essere utilizzati per interagire con il sistema
operativo, il sistema operativo può disabilitare gli interrupt.

\subsection{Frammentazione}
La frammentazione interna è composta dalla memoria allocata per un programma,
ma non utilizzata.

La frammentazione esterna invece è la memoria che non viene utilizzata perchè
non abbastanza grande da ospitare un processo e tra due address spaces.

Il \textit{compattamento} della memoria non è una soluzione attuabile in quanto
sarebbe estrememante inefficiente.

Esistono diversi metodi per poter utilizzare le porzioni di memoria
frammentata.

\subsubsection{Segmentazione}
La segmentazione è utilizzata per poter utilizzare la memoria frammentata
esternamente.
Si utilizza la \textit{rilocazione dinamica}, ovvero la traduzione di indirizzi
logici in indirizzi fisici attraverso una funzione di rilocazione implementata
nell'hardware, \textbf{MMU}.


\paragraph{Registri base limite}
Nell' $8086$ in $16bit$ la segmentazione era classificata in:
\begin{itemize}
  \item \textbf{CS:} code segment;
  \item \textbf{DS:} data segment;
  \item \textbf{SS:} stack segment;
  \item \textbf{ES:} extra segment.
\end{itemize}

In questa versione, la \textbf{MMU} mantiene una lista di registri \textit{
base/limite}, ad ogni coppia corrisponde un segmento di memoria frammentata,
la base contiene il primo indirizzo di memoria utile, mentre il limite quanta
memoria contigua è disponibile.

Nel momento in cui il processore genere un indirizzo virtuale prende in
considerazione questi registri; l'indirizzo viene generato in fase di \textit{
fetch} dell'istruzione.
Se l'indirizzo generato supera il valore contenuto nel registro limite, un
eccezione viene generata in quanto l'indirizzo richiesto non apparterrebbe al
segmento.

\paragraph{Tabelle dei descrittori}
L'architettura $386$, risolve due problemi relativi alla precedente soluzione:
una più veloce risoluzione dei permessi ed il limite di registri salvabili
in MMU.


Per ottenere l'indirizzo di memoria giusto viene ora utilizzata una parola che
indica sia il \emph{segmento} di riferimento che l'offset all'interno del
segmento (indirizzo virtuale).


Inoltre viene introdotta una struttura dai tabellare, mantenuta all'interno
della memoria riservata del sistema operativo, detta \textit{tabella dei
descrittori}, contenente i dati relativi alla coppia dei registri \textit{base/
limite} di ogni segmento, il bit di protezione e la gestione dei privilegi
DPL).
La MMU conterrà soltanto due registri, \textbf{STLR} (Segment Table Limit
Register) la dimensione della tabella dei segmenti e \textbf{STBR} (Segment
Table Base Register) l'indirizzo base della tabella dei segmenti.

\begin{itemize}
  \item \textbf{CS:} code segment, con una local descriptor table;
  \item \textbf{DS:} data segment, con una global descriptor table;
  \item \textbf{SS:} stack segment;
  \item \textbf{ES:} extra segment.
\end{itemize}
\[
  MAX(CPL \quad RPL) \leq DPL \quad BASE + OFFSET
\]

I segmenti di codice (\textbf{CS}) possono essere utilizzati da più processi
uguali.

Quindi le funzioni \textit{Jmp, Call e Ret} per accedere ad un segmento
utilizzano due dati: \textbf{offset} e \textbf{selettore offset}.

Definiamo \textbf{EIP} (Instruction Pointer) e \textbf{ESP} (Stack Pointer),
rispettivamente appartengono ad un determinato \textbf{CS} (Code Segment) e
\textbf{SS} (Stack Segment).

\textbf{EIP} e \textbf{CS} alla chiamata di \textit{Jmp, Call o Ret}, verranno
salvati nello stack (\textbf{ESP}).



Il \textbf{Call Gate} è uno strato che si aggiunge all'indirizzo virtuale per
non permettere l'erroneo accesso a sezioni di memoria non utilizzabili dall'
utente o dal programma.


Ogni livello di privilegio ha uno \textbf{stack separato}, per offrire una
maggiore stabilità del sistema, quindi se una \emph{Call} cambia il livello
di privilegio dovrà anche cambiare lo \emph{stack}: se il livello di privilegio
si riduce, tutti i registri che contengono riferimenti a sezioni di memoria
privilegiata vengono azzerati.

\paragraph{Interrupt}
Se l'utente è kernel, al momento dell'arrivo di un interrupt non c'è bisogno
di cambiare lo stack, altrimenti sì.

All'arrivo di un interrupt \textbf{eflag} (flag dell'interrupt) viene salvato,
quindi l'interrupt viene disattivato, \textbf{iret} esce dall'interrupt
handler.


\subsection{Paginazione}
Per il processo la paginazione è trasparente.
La memoria fisica è divisa equamente in \textbf{page frame}, la
virtualizzazione corrispondente sono le \textbf{pagine}, ovvero le pagine sono
ordinate, mentre la rappresentazione nel \textbf{page frame} non è
necessariamente ordinata, ma i \textbf{page frame} sono numerati.
Le pagine sono grandi $4096byte$, in generale sempre potenze di $2$.


Quindi si utilizza una tabella di appoggio, lunga tanto quanto il numero di
pagine, per associare \emph{page frames} a \emph{pagine}, in questa tabella la
colonna \textbf{present}(flag), vero se la pagina è utilizzata nella memoria
fisica.
Una tabella simile è creata ed associata per ogni processo, la tabella è
creabile e modificabile solo dal kernel.


Dato un indirizzo virtuale si divide per la grandezza della pagina per ottenere
il numero della pagina, quindi dalla tabella di appoggio si seleziona il \emph{
page frame} corrispondente, quindi si utilizza il resto della divisione per
determinare l'indirizzo di \textbf{memoria fisica} corrispondente.
\[
  \text{Indirizzo fisico} = \text{Indirizzo virtuale}/3 * \text{Grandezza delle
  pagine} + \text{Indirizzo virtuale} \% 3 
\]

Stabilita la grandezza della pagina si interpretano i primi $bit$ dell' 
indirizzo virtuale per trovare il numero di pagina, mentre il restante si usa
come \emph{offset}, dato l'indirizzo della base si appende l'offset per
ottenere l'indirizzo fisico.


La paginazione usa appunto le \textbf{paging tables}, per ogni riga abbiamo:
\begin{itemize}
  \item il \textbf{PPN} (Physical Page Number);
  \item un bit di validità;
  \item un bit di presenza (se la pagina è presente);
  \item un bit di accesso (se ha già acceduto);
  \item un bit di modifica;
  \item dei bit per il livello di privilegio (r,w,x).
\end{itemize}


Per migliorare l'efficienza si utilizza una cache, la \textbf{TLB}
(Translation Lookaside Buffer), da un \textbf{VPN} (Virtual Page Number),
si ottiene il \textbf{PPN} (Physical Page Number) ed il \textbf{PFN}
(Physical Frame Number).


Per migliorare ulteriormente la paginazione, non si utilizzano i bit di
accesso e modifica, sarà quindi il kernel a tenere traccia di questi valori,
lo stesso ragionamento si può fare per i bit sui permessi.

\paragraph{Osservazione}
Per indiciare una ram di $4GB$ sono necessari $32bit$, un indirizzo virtuale
è composto da $20bit$ per selezionare la pagina e:
\begin{itemize}
  \item $v=1bit$;
  \item $p=1bit$;
  \item $r/w/x=3bit$;
  \item $d=1bit$.
\end{itemize}

Quindi abbiamo un totale di $27bit$ ($32bit$ normalizzati), per una RAM da
$4GB$ la \emph{paging table} occuperebbe $400mb$ di memoria contigua nella
RAM.

Per rendere più efficiente la tabella, i $20bit$ di indirizzamento di dividono
logicamente in 2 blocchi da $10bit$, il primo per indirizzare il blocco di
memoria, il secondo per ottenere l'effettivo indirizzo fisico.

Il primo blocco in realtà indirizza su di una nuova tabella di appoggio il cui
contenuto sarà se la pagina è occupata o no e l'indirizzo della tabella
corrispondente, quindi si usa la tabella di appoggio ed il secondo blocco di
$10bit$ per ottenere l'indirizzo fisico.

\subsubsection{Base limite, Segmentazione e paginazione}
Compariamo i sistemi di suddivisione della memoria: \textbf{B+L}, \textbf{
segmentazione} e \textbf{paginazione}.

B+L:
\begin{itemize}
  \item $?$ spazio di indirizzamento contiguo in \emph{RAM};
  \item $-$ frammentazione interna;
  \item $-$ frammentazione esterna;
  \item $-$ condivisione codice;
  \item $+$ isolamento\\protezione;
  \item $-$ efficiente;
  \item $-$ spazio di indirizzamento contiguo.
\end{itemize}

Segmentazione:
\begin{itemize}
  \item $?$ spazio di indirizzamento contiguo nel segmento, ma non nel processo
    ;
  \item $-$ frammentazione interna;
  \item $-$ frammentazione esterna;
  \item $\mp$ condivisione codice;
  \item $+$ isolamento\\protezione;
  \item $\pm$ efficiente;
  \item $-$ spazio di indirizzamento contiguo.
\end{itemize}

Paginazione:
\begin{itemize}
  \item $-$ spazio di indirizzamento non contiguo;
  \item $\pm$ frammentazione interna ($4k$);
  \item $-$ frammentazione esterna;
  \item $\mp$ condivisione codice;
  \item $+$ isolamento\\protezione;
  \item $\pm$ efficiente;
  \item $-$ spazio di indirizzamento contiguo.
\end{itemize}

\section{System calls}
\subsection{fork}
La funzione \emph{fork} crea un processo quasi identico al processo padre e
restituisce un numero positivo chiamato \emph{pid\_t}.
Ogni processo ha un \emph{ppid} ( \emph{pid} del padre).

\subsection{exit}
La funzione \emph{exit} ritorna $0$ se il processo si chiude correttamente,
altrimenti un numero.

\subsection{wait}
Ritorna un \emph{pid\_t} quando il processo passato come argomento termina. 

\subsection{exec}
Sostituisce il processo corrente con uno nuovo, l'argomento è un PATH.

\subsection{open}
Apre un file e gli assegna il primo file descriptor disponibile.

\subsection{dup}
Duplica un fd creandone uno nuovo che punta allo stesso oggetto.

\subsection{dup2}
La dup2 ha due parametri, il primo il vecchio file descriptor ed il secondo il
nuovo, duplica il primo parametro nel secondo, fa una close del secondo
parametro nel caso fosse già aperto.

\subsection{pipe}
\begin{lstlisting}[
  language=bash,
  escapeinside={(£}{£)},
  caption={
    Esempio di pipe (|)
  }
]
# (1)
ls -l | grep cammello

# (2)
ls -l > camelus
grep cammello < camelus
\end{lstlisting}

L'operatore $|$ permette di chiamare la syscall \textit{pipe()} la quale
permette di eseguire l'operazione esclusivamente in memoria, difatti nel
secondo esempio \textit{camelus} sarebbe scritto su disco e, su risultati
molto grandi sarebbe molto più lento ed inefficiente.

Quindi \textit{pipe(fds)} prende come argomento un array di file descriptors
ed utilizza un buffer riservato al kernel per svolgere la lettura.

Il funzionamento per il  comando $(1)$ sarà:
\begin{itemize}
  \item il processo padre chiama la \textit{pipe()}, $2$ \textit{fork()},
    \textit{close(3)}, \textit{close(4)} e $2$ \textit{wait()};
  \item la prima figlia \textit{ls -l} chiama:
    \begin{itemize}
      \item \textit{close(1)};
      \item \textit{dup(3)};
      \item \textit{close(2)};
      \item \textit{close(4)} dove $4$ è il fd della pipe;
      \item \textit{exec(...)};
    \end{itemize}
  \item la seconda figlia \textit{grep cammello} chiama:
    \begin{itemize}
      \item \textit{close(0)};
      \item \textit{dup(4)}, dove $3$ è il file descriptor della fork;
      \item \textit{close(2)};
      \item \textit{close(4)} dove $4$ è il fd della pipe;
      \item \textit{exec(...)};
    \end{itemize}
\end{itemize}

La shell è il programma che attende la scrittura e l'invio dei comandi, una
delle funzionalità è di espandere le variabili passate (es. *.c).

\section{Scheduling}
Lo scheduler è un programma che implementa un algoritmo di scheduling il quale,
dato un insieme di richieste di accesso ad una risorsa, stabilisce un
ordinamento temporale per l'esecuzione di tali richieste, privilegiando quelle
che rispettano determinati parametri secondo una certa politica di scheduling,
in modo da ottimizzare l'accesso a tale risorsa e consentire così
l'espletamento del servizio/istruzione o processo desiderato.

Funziona tramite i meccanismi di \emph{LDE} (user, kernel) e di context
switching.

Ricapitolando, il ciclo di vita di un processo:
\begin{itemize}
  \item init;
  \item \emph{EMBRYO};
  \item fork;
  \item ready;
  \item running;
  \item zombie.
\end{itemize}

\subsection{Turn-around Time}
\[
  T_{TA}=T_{ \text{Completion}}-T_{ \text{Arrival}}
\]

\subsection{Response Time}
\[
  T_ { \text{RESP}}=T_{ \text{First Run}}-T_{ \text{Arrival}}
\]
