\chapter{Introduzione ai sistemi operativi}
\begin{figure}[ht]
    \centering
    \incfig{struttura}
    \caption{Struttura base di un calcolatore}
    \label{fig:struttura}
\end{figure}

Lo scopo di questa sezione è di virtualizzare le componenti fisiche di un
calcolatore per renderne più facile l'utilizzo.

Sinonimi di \emph{sistema operativo} sono \textbf{macchina virtuale} (virtual
machine) e \textbf{gestore dlle risorse} (resource manager).

\section{Virtualizzazione della CPU}
L'unica nota in questo programma è l'utilizzo della funzione \emph{spin()}, che
controlla l'orologio di sistema fino a quanto $1$ minuto non è passato, quindi
riprende l'esecuzione.

\lstinputlisting[caption=Interazione con la  CPU usando la funzione spin()]{listings/cpu.c}

Da notare che se il programma viene eseguito con un comando del tipo:
\begin{lstlisting}[language=bash,caption=Esecuzione in concorrenza]
prompt> ./cpu A & ./cpu B &./cpu C & ./cpu D
[7353]
[7354]
[7355]
[7356]
A
B
D
C
A
B
D
C
A
...
\end{lstlisting}

Possiamo osservare che sembra che vengano eseguiti simultaneamente quattro
programmi diversi, ciò è dovuto alla \emph{virtualizzazione della CPU}.

\section{Memoria virtualizzata}
\lstinputlisting[caption=Dimostrazione dell'utilizzo di indirizzi virtuali]{listings/mem.c}

La memoria \textbf{RAM} è generalmente virtualizzata, difatti se eseguiamo
il programma precedente con un comando tipo:
\begin{lstlisting}[language=bash,caption=Esecuzione in concorrenza]
prompt > ./mem &; ./mem &
[1] 24113
[2] 24114
(24113) address pointed by p: 0x200000
(24114) address pointed by p: 0x200000
(24113) p: 1
(24114) p: 1
(24113) p: 2
(24114) p: 2
(24113) p: 3
(24114) p: 3
(24113) p: 4
(24114) p: 4
...
\end{lstlisting}

Possiamo osservare il funzionamento della virtualizzazione della memoria,
infatti ogni programma è lanciato in un proprio \textbf{virtual address space};
semplifica la gestione della memoria per il programma ed è mappato ad una
sezione della memoria fisica.

\section{Periferiche}
Nell'interazione di un calcolatore con le periferiche, è il kernel che
interagisce direttamente con l' hardware, mentre le applicazioni possono
utilizzare le periferiche solo attraverso il kernel.


Vi sono $4$ diversi livelli di privilegi per interagire con i dispositivi:
\begin{itemize}
  \item $0$, il livello più privilegiato, utilizzato dal kernel;
  \item $1$ solitamente non utilizzato;
  \item $2$ solitamente non utilizzato;
  \item $3$ il livello meno provilegiato, utilizzato da tutti gli utenti
    compreso l'utente di \emph{root}.
\end{itemize}

Se l'utende richiede un interazione con una periferica, un comando di
\emph{TRAP} viene inviato al kernel, il quale restituise un comando di
\emph{return from TRAP}.

\section{Spazi di indirizzamento}
Ad ogni processo in esecuzione corrisponde un \emph{address space}, una
virtualizzazione della memoria che comporta diversi vantaggi, tra cui non
permettere ai programmi di modificare memoria non nella propria \emph{address
space} ed in generale un efficienza maggiore.

Lo spazio di indirizzamento non può essere più grande della memoria fisica ed
occupa una sezione di memoria contigua.

Utilizzando questo metodo abbiamo due problemi di frammentazionem interna ed
esterna.


Un processo è composto da:
\begin{itemize}
  \item \textbf{codice:} statico, regione fissa;
  \item \textbf{dati:} regione fissa;
  \item \textbf{stack:} regione dinamica, può crescere o diminuire;
  \item \textbf{heap:} regione dinamica.
\end{itemize}


Quindi l'address space è composto, in ordine, da:
\begin{itemize}
  \item \textbf{codice del programma:} dove risiedono le istruzioni;
  \item \textbf{heap:} contiene i dati creati con i \emph{malloc} e le
    strutture dinamiche, cresce verso il basso;
  \item \textbf{memoria libera:} che può essere occupata dallo heap e dallo
    stack;
  \item \textbf{stack:} dove risiedono le variabili locali, i valori di ritorno
    delle funzioni, etc. e cresce verso l'alto.
\end{itemize}


All'interno del sistema operativo è contenuta una struttura dati che contiene
informazioni relative ai processi in memoria, \textbf{PCB} (Process Control
Bloc).

\subsection{Virtualizzazione}
Gli obiettivi della virtualizzazione sono: trasparenza, efficienza e protezione
ovvero un programma non deve accorgersi che sta utilizzando solo una porzione
della memoria disponibile.


Per implementare queste funzioni si utilizza una \textbf{MMU}; al suo interno
salviamo un registro chiamato base, il quale cambia a seconda del programma
in esecuzione.

Questa implementazione non è corretta in quanto non abbiamo indicato un limite
al valore che possiamo assegnare alla base, quindi i processi potrebbero
accedere a memoria successiva alla loro, per cui sono gestiti anche i limiti
su quali valori posso assegnare al registro base.

Nel caso di un \textit{segmentation fault} (errore relativo all'accesso di
memoria non propria) vengono sollevati gli \textbf{interrupt handlers}, e gli
\textbf{exception handlers}, i quali gestiscono le eccezioni.
Altri errori simili che sollevano un interrupt sono la divisione per $0$.

Gli interrupt possono anche essere utilizzati per interagire con il sistema
operativo, il sistema operativo può disabilitare gli interrupt.

\subsection{Frammentazione}
La frammentazione interna è composta dalla memoria allocata per un programma,
ma non utilizzata.

La frammentazione esterna invece è la memoria che non viene utilizzata perchè
non abbastanza grande da ospitare un processo e tra due address spaces.

\subsubsection{Segmentazione}
La segmentazione è utilizzata per poter utilizzare la memoria frammentata
esternamente, per cui offre la possibilità di creare uno spazio di
indirizzamento in memoria non contigua.


Nell' $8086$ in $16bit$ la segmentazione era classificata in:
\begin{itemize}
  \item \textbf{CS:} code segment;
  \item \textbf{DS:} data segment;
  \item \textbf{SS:} stack segment;
  \item \textbf{ES:} extra segment.
\end{itemize}

Nel $386$, introduce le tabelle dei descrittori, che descrivono i segmenti che
che includon, quindi contengono dati per: base, limite, bit di protezione e
privilegi (DPL).

\begin{itemize}
  \item \textbf{CS:} code segment, con una local descriptor table;
  \item \textbf{DS:} data segment, con una global descriptor table;
  \item \textbf{SS:} stack segment;
  \item \textbf{ES:} extra segment.
\end{itemize}
\[
  MAX(CPL \quad RPL) \leq DPL \quad BASE + OFFSET
\]

I segmenti di codice (\textbf{CS}) possono essere utilizzati da piùù processi uguali.
