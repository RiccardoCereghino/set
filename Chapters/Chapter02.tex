\chapter{Introduzione ai sistemi operativi}
\begin{figure}[ht]
    \centering
    \incfig{struttura}
    \caption{Struttura base di un calcolatore}
    \label{fig:struttura}
\end{figure}

\section{CPU virtualizzata}
\lstinputlisting[caption=Dimostrazione della virtualizzazione della CPU]{listings/cpu.c}
\section{Memoria virtualizzata}
\lstinputlisting[caption=Dimostrazione dell'utilizzo di indirizzi virtuali]{listings/mem.c}

\section{Periferiche}
Nell'interazione di un calcolatore con le periferiche, è il kernel che
interagisce direttamente con l' hardware, mentre le applicazioni possono
utilizzare le periferiche solo attraverso il kernel.


Vi sono $4$ diversi livelli di privilegi per interagire con i dispositivi:
\begin{itemize}
  \item $0$, il livello più privilegiato, utilizzato dal kernel;
  \item $1$ solitamente non utilizzato;
  \item $2$ solitamente non utilizzato;
  \item $3$ il livello meno provilegiato, utilizzato da tutti gli utenti
    compreso l'utente di \emph{root}.
\end{itemize}

Se l'utende richiede un interazione con una periferica, un comando di
\emph{TRAP} viene inviato al kernel, il quale restituise un comando di
\emph{return from TRAP}.
