\section{Networking}
Vengono effettuate delle syscall, per eseguire operazioni necessario all'invio di dati.
Per esempio la syscall send, copia dalla ram una certa sezione di memoria nel NIC,
che verrà inviata in rete.
???

\subsection{TCP/UDP}
Il protocollo TCP garantisce l'invio del messaggio, dato che attende una risposta
da parte del ricevente.

Il protocollo UDP invia datagrammi, ma non si è certi se la ricezione è avvenuta.
\subsubsection{Device driver}
E' un modulo che gestisce le rusirse fusuche dei vari moduli, tra cui NIC,
gestiscono le interruzioni del dispositivo.

\subsection{Invio}
Il DMA presente nel NIC è il componente che accede alla memoria.

L'unità di gestione della memoria MMU,
permette di utilizzare indirizzi virtuali in indirizzi fisici,
quindi mentre il processore utilizza indirizzi virtuali,
il DMA del NIC utilizzerà indirizzi fisici.

Interrupt, buffer
\subsection{Ricezione}
Per la ricezione di pacchetti devono essere sempre disponibili dei buffer,
su cui NIC può scrivere una volta ricevuto il pacchetto.

Il pacchetto in arrivo viene processato dal NIC quindi con il DMA inserito nel buffer,
quindi un interrupt crea nuovi buffer per possibili nuovi messaggi,
quindi il sistema legge il buffer copiandolo nel buffer dell'applicazione,
direzionando quindi l'output verso l'utente corretto,
quindi il buffer di ricezione può essere riutilizzato per nuovi pacchetti.

Il comportamento standard di un applicazione che implementa networking è un
meccanismo bloccante per cui l'applicazione rimane in attesa fintanto che il
buffer non è stato scritto al suo interno.

\section{trasmissione di datu su bus ring based}
La problematica principale del DMA di tipo bus mastering è che deve essere
programmato in anticipo, ovvero è limitata da quanti buffer sono stati programmati
per essere utilizzati dal DMA ed è dipendente dalla velocità del processore a
liberare il buffer, per cui fintanto che il DMA è occupato, verrano persi i
dati nel frattempo ricevuti dalla rete.


Quindi si implementa una struttura dati più complicata, integrando una piccola
CPU all'interno della NIC (ring based).
\subsection{Ring}
Il ring è composto da un anello contenente più di un buffer, e due puntatori,
un puntatore di inizio ed uno di fine gestiti da due processori diversi;
la CPU si occupa di aggiungere elementi all'interno del ring quindi gestirà
l'indice di fine, il puntatore al primo elemento viene gestito dal processore
all'interno della NIC.
SI ottiene una coda di buffer che si possono aggiungere o togliere a seconda
se si sta gestendo un invio o ricezione di dati.

Quindi la NIC legge i buffer in memoria e usa un flag per impostare se il buffer
è libero oppure se deve essere processato.
Quindi il'unica comunicazione tra processore e DMA è la comunicazione della 
quantità di buffer che dovranno essere utilizzati al DMA.
